\RequirePackage{filecontents}

\begin{filecontents}{\jobname.bib}
\end{filecontents}

\documentclass{article}
\usepackage[utf8]{inputenc}
\usepackage[T1]{fontenc}
\usepackage[french]{babel}
\usepackage[parfill]{parskip}
\usepackage{amsmath}
\usepackage{amssymb}
\usepackage{amsfonts}
\usepackage{subfigure}
\usepackage[font={small}]{caption}
\usepackage{float}
\usepackage{listingsutf8}
\usepackage{fullpage}
\usepackage[nochapter]{vhistory}
\usepackage{hyperref}
\usepackage{titlesec}
\usepackage{xcolor}
\usepackage{verbatim}
\usepackage{graphicx}
\usepackage{subcaption}
\usepackage{comment}

\usepackage{natbib}
\usepackage{url}
\usepackage{algpseudocode}

\usepackage{adjustbox}



\newcommand*{\MyIncludeGraphicsMaxSize}[2][]{%
    \begin{adjustbox}{max size={\textwidth}{\textheight}}
        \includegraphics[#1]{#2}%
    \end{adjustbox}
}
\usepackage{array,booktabs,ragged2e}
\usepackage{lstdoc}
\newcolumntype{R}[1]{>{\RaggedLeft\arraybackslash}p{#1}}
\newcolumntype{D}[1]{>{\RaggedLeft\arraybackslash}p{#1}}

% -----------------------------------------------------
% -----------------------------------------------------
% -----------------------------------------------------

\hypersetup{
%couleurs des liens cliquable changée pour une meilleur lisibilité
    colorlinks=true,
    linkcolor=blue,
    filecolor=magenta,
    urlcolor=cyan,
    pdfpagemode=FullScreen,
}


\title{R-Trees}
\author{Akajou Ilias, Noé Bourgeois }
\date{March 2023}

\begin{document}

    \maketitle
    \tableofcontents
    \newpage

    \section{R-Trees}
    \subsection{Contexte}
    Ce projet consiste en

    \href{https://en.wikipedia.org/wiki/R-tree}{https://en.wikipedia.org/wiki/R-tree}


    La recherche est définie de la façon suivante :
    — Pour une feuille :
    — Si le point appartient au MBR du nœud et le point appartient au polygone,
    retourner "this",
    — Sinon, retourner "null" ;
    — Pour un nœud :
    — Si le point appartient au MBR, tester récursivement l’appartenance pour chacun
    des sous-nœuds. Dès qu’un appel récursif renvoie autre chose que null, retourner
    ce résultat,
    — Sinon, retourner null ;
    — Pour un arbre : appeler la fonction de recherche sur la racine.

    \subsubsection{Point In Polygon}
    \subsubsection{Minimum Bounding Rectangle}


    \subsection{Structure}

    \subsection{algorithme}
    \label{sec:}

    \subsubsection{Description}


    \subsubsection{Complexité}
    \label{sec:Complexite}

    Temps :
    \begin{itemize}
        \item Borne inférieure :
        \Omega{\displaystyle (\vert V\vert +\vert E\vert )}

        \item Pire cas :
            {\displaystyle {\mathcal {O}}(\vert V\vert +\vert E\vert )}

    \end{itemize}



    Espace :
    \begin{itemize}
        \item Borne inférieure :
        $\Omega{\displaystyle (\vert V\vert +\vert E\vert )}$

        \item Pire cas :
            {\displaystyle 2\vert E\vert +{\mathcal {O}}(\vert V\vert )}

    \end{itemize}
    \subsubsection{Mise en oeuvre en Java }

    \begin{lstlisting}[language=java]
    \end{lstlisting}

    \newpage

    \subsubsection{Minimum Bounding Rectangle}
    \subsection{Introduction}


    \subsubsection{Description}
  \\
    \begin{lstlisting}[language=java]

    \end{lstlisting}

    \subsubsection{Complexité}
    \newpage

    \subsection{Création}
    \subsubsection{Split quadratique}
    \label{sec:Quadratique_introduction}

    \subsubsection{algorithme}


    \subsubsection{Description}

    \subsubsection{Complexité}

    \subsubsection{Mise en oeuvre en Java }

    \begin{lstlisting}[language=java]
    \end{lstlisting}


    \subsubsection{Split linéaire}
    \label{sec:Lineaire_introduction}

    \subsubsection{algorithme}


    \subsubsection{Description}

    \subsubsection{Complexité}

    \subsubsection{Mise en oeuvre en Java }

    \begin{lstlisting}[language=java]
    \end{lstlisting}


    \newpage

    \section{Expériences sur des données réelles }
    \subsection{La librairie Geotools}
    \subsubsection{Découpe de la Belgique en secteurs statistiques}
    \href{https://statbel.fgov.be/fr/open-data?category=191}{https://statbel.fgov.be/fr/open-data?category=191}
    \subsubsection{Pays du monde}
    \href{https://datacatalog.worldbank.org/search/dataset/0038272/World-Bank-Official-Boundaries}{https://datacatalog.worldbank.org/search/dataset/0038272/World-Bank-Official-Boundaries}
    \subsubsection{Communes françaises}
    \href{https://www.data.gouv.fr/fr/datasets/contours-des-regions-francaises-sur-openstreetmap/}{https://www.data.gouv.fr/fr/datasets/contours-des-regions-francaises-sur-openstreetmap/}
    \begin{table}
        \centering

        \begin{tabular}{|c|c|c|c
        |c|c|c|c}
            \hline
%\multicolumn{1}{r}{\textbf{graphe}} & \textbf{nœuds} & \textbf{arêtes} & \textbf{CCM} \\
            graphe & nœuds & arêtes & CCM  \\
            \hline
            1v & 1 & 0 & 1\\
            2v1e & 2 & 1 & 1\\
            graphtest & 4 & 5 & ~0.6 \\
            ego-facebook & 4039 & 88234 & 0.6055\\
            roadNet-PA & 1088092 & 1541898 & 0.046\\
            roadNet-CA & 1971281 & 5533214 & 0.0464\\
            com-LiveJournal & 4036538 & 34681189 & 0.2843\\
            com-friendster & 124833781 & 1806067135 & 0.1623\\
            \hline
        \end{tabular}
        \caption{\label{tab:table}Données}
    \end{table}

%\newpage

%\begin{comment}


    \begin{table}
        \centering
        \begin{tabular}{|c|c|c|c|c
        |c|c|c|c|c}
            \hline
%\multicolumn{1}{r}{\textbf{graphe}} & \textbf{nœuds} & \textbf{arêtes} & \textbf{CCM} \\
            graphe & dégénérescence & TMCD & colorabilité & TMCC \\
            \hline
            1v & 0 & 0.7719 & 1 & 0,5106 \\
            2v1e & 1 & 0.7013 & 2 & 0,4722 \\
            graphtest & 2 & 0.9016 & 3 & 2,828\\
            ego-facebook  & 115 & 80 & 75 & 61\\
            roadNet-PA  & 6 & 610 & 4 & +/-1740000 \\
            roadNet-CA  & 6 & 1741 & 5 & +/-5400000  \\
            com-LiveJournal  & 360 & 69784 & / & /  \\
%com-friendster   & / & / & / & / \\
            \hline
        \end{tabular}
        \caption{\label{tab:table-name}Résultats }
    \end{table}

%\end{comment}


    \newpage

    \subsection{Conclusions }



    \newpage



    \section{Ressources}
    \underlined{(cf. énoncé)}

    Wave Function Collapse Coloring: A New Heuristic for Fast Vertex Coloring :

    \href{https://www.researchgate.net/publication/354088959_Wave_Function_Collapse_Coloring_A_New_Heuristic_for_Fast_Vertex_Coloring}{https://www.researchgate.net/publication/\\354088959_Wave_Function_Collapse_Coloring_A_New_Heuristic_for_Fast_Vertex_Coloring}

    Théorème de Brook :

    \href{https://www.sciencedirect.com/science/article/pii/0095895675900891?via%3Dihub}{https://www.sciencedirect.com/science/article/pii/0095895675900891?via%3Dihub
    }

    Preuve du Lemme : \href{https://math.stackexchange.com/questions/4430669/proof-suggestion-for-the-chromatic-number-of-a-k-degenerate-graph-is-inferior}{https://math.stackexchange.com/questions/4430669/proof-suggestion-for-the-chromatic-number-of-a-k-degenerate-graph-is-inferior}

    Rédaction scientifique:

    \href{http://informatique.umons.ac.be/algo/redacSci.pdf.}{http://informatique.umons.ac.be/algo/redacSci.pdf.
    }

    Ressources bibliographiques:

    \href{https://www.bibtex.com/.}{https://www.bibtex.com/.}


    Classes de la bibliothèque Java
    algs4.jar, disponible à l’adresse suivante :

    \href{https://algs4.cs.princeton.edu/code/.}{https://algs4.cs.princeton.edu/code/.}

    L'écriture du code a été accélérée à l'aide du plugin "Github Copilot"
    \href{https://copilot.github.com/}{https://copilot.github.com/}


\end{document}